\section{Introduction}

The ventral nerve cord (VNC) is a major structure of the invertebrate central nervous system, including that of \textit{Drosophila}.
It runs down the ventral plane of the organism, and is the insect motor control center.
It contains both descending neurons (DNs) and ascending neurons (ANs).
The former have their bodies and dendrites in the brain, and their axons projecting through the VNC, while it is the reverse for ANs.

\vspace{\baselineskip}

It has been theoretically hypothesized that artificially driving a small set of DNs could flexibly elicit multiple complex behaviors in individuals~\cite{ijspeert2007}, like swimming and walking.
In practice, the hypothesis has not been testable before transgenic strains of \textit{Drosophila} were made available in 2014~\cite{bidaye2014} and 2018~\cite{namiki2018}, allowing for DNs neuron targeting for studies of anatomy and function.
It has then been proven that individual DNs activation can trigger distinct behavior like backward walking, in experiments on single DN stimulation.

\vspace{\baselineskip}

More recently, technological advances such as two-photon imaging have allowed the simultaneous recording of multiple neurons at once~\cite{chen2018}.
Using this technique, an experiment consisting of the simultaneous recording of 123 DNs of \textit{Drosophila} was run, while the subject flies were freely moving on an air-suspended spherical treadmill.
The data gathered thus consist of spontaneous behavioral observations (video recordings), associated with the respective neural data (two-photon imaging of DNs) and kinematic data (relative joint positions and angles of leg segments).

\vspace{\baselineskip}

Because the data gathered is simultaneous, it allows to try to link specific behaviors with specific neural activity.
It is an observational study, though, so that causation cannot be inferred, only correlation (if present).
Thus, the goal of this project is to determine possible correlations between DNs of \textit{Drosophila} and some high level behaviors of the insect, and see if classification allows for behavioral prediction.

\newpage
