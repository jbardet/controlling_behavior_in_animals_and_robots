\pdfoptionpdfminorversion=7
\documentclass[
	10pt, % font size (default is 10pt)
	english, % language
	twoside % documents can be either one sided (eg articles) or two sided (eg books), affects margins and stuff
]{article}

\def \title {Controlling Behavior in Animals and Robots}
\def \subtitle {Mini-project}
\def \author {James Bardet -- Nicolas Furrer -- Claire Meyer}
\def \creator {Claire Meyer}
\def \institution {EPFL}
\def \professor {Pavan Ramdya}
\def \course {Controlling Behavior in Animals and Robots}
\def \place {Écublens}
\def \subject {bioscience}
\def \sciper {287045 -- 215236 -- 204326}
\def \tags {project}


%%%%%%%%%%%%%%%%%%%%%%%%%%%%%%%%
%% PLATFORM DEPENDANT CONDITIONS
%%%%%%%%%%%%%%%%%%%%%%%%%%%%%%%%

\usepackage{ifplatform}
% allows to detect which os is running the compilation

%%%%%%%%%%%%%%%%%%%%%%%
%% INTERNATIONALIZATION
%%%%%%%%%%%%%%%%%%%%%%%

\usepackage{babel}
% allows the internationalization of the document, language specified in the document class
% should be placed soon after the \documentclass command, so that all the other packages loaded afterwards know the used language

\usepackage[T1]{fontenc} 
% defines the font encoding
% is used when in need of non-english caracters 
% (if the document is only english, then no need for it, the default font encoding is good)
% allows for automatic hyphenation of accentuated words
% allows for successful search of accentuated words in pdf
% allows for successful copy pasting of accentuated words from pdf

\usepackage[utf8]{inputenc}
% defines the character encoding


%%%%%%%%%%%%%%
%% MATHEMATICS
%%%%%%%%%%%%%%

\usepackage{mathtools}
% simplifies the usage of complicated mathematical formulae
% loads the amsmath package, fixes some of its quirks and adds some useful settings, symbols, and environments to it

\setcounter{MaxMatrixCols}{20}

\usepackage{siunitx}

\usepackage{empheq}


%%%%%%%
%% TEXT
%%%%%%%

\usepackage{textcomp}
% provides extra symbols, e.g. arrows like \textrightarrow, various currencies (\texteuro,...), things like \textcelsius and many others

\usepackage{alltt}
% processes text as if by a typewriter (verbatim), BUT still recognizes commands

\usepackage{cprotect}


%%%%%%%%%%%%%
%% HYPERLINKS
%%%%%%%%%%%%%

\usepackage[
	bookmarks=true, % shows the bookmarks bar when displaying the document
	bookmarksnumbered=false, % put section numbers in bookmarks
	pdftoolbar= true, % shows or hide Acrobat?s toolbar
	pdfmenubar=true, % shows or hide Acrobat?s menu
	unicode=true, % allows to use characters of non-Latin based languages in Acrobat?s bookmarks
	%%%%%%%%%%%%%%%%%
	colorlinks=true, % surround the links by color frames (false) or colors the text of the links (true)
	linkcolor=blue, % color of internal links (sections, pages, etc.)
	citecolor=blue, % color of citation links (bibliography)
	filecolor=blue, % color of file links
	urlcolor=blue, % color of URL links (mail, web)
	anchorcolor=black, % color of anchor text
	menucolor=black, % color for menu links
	linktoc=section, % defines which part of an entry in the table of contents is made into a link (none, section, page, all)
	%%%%%%%%%%%%%%%%%
	pdffitwindow=false, % resizes document window to fit document size
	pdfnewwindow=true, % define if a new PDF window should get opened when a link leads out of the current document, the option is ignored if the link leads to an http/https address
	pdfstartpage={1}, % page at which PDF document opens
	pagebackref=false, % activate back references inside bibliography, must be specified as part of the \usepackage{} statement, must be set to false to work with biblatex
	%%%%%%%%%%%%%%%%%
	pdftitle={\subtitle}, %	defines the title that gets displayed in the "Document Info" window of Acrobat
	pdfauthor={\author}, % the name of the PDF?s author, works like the one above
	pdfsubject={\subject}, % subject of the document, it works like the one above
	pdfcreator={\creator}, % creator of the document, it works like the one above
	pdfproducer={TeXstudio}, % producer of the document, it works like the one above
	pdfkeywords={\tags}, % list of keywords, separated by commas, example below
]{hyperref}
% provides LaTeX the ability to create hyperlinks within the documen


%%%%%%%%%%%%%
%% APPEARANCE
%%%%%%%%%%%%%

\usepackage[
dvipsnames, % defines the use of the colors from the dvips / SVG set
table, % allows the use of colors in tables
]{xcolor}
% provides both foreground and background color management, has to be loaded before pdfpages to avoid option clash

\usepackage{pdfrender}



%%%%%%%%%%%%
%% PDF PAGES
%%%%%%%%%%%%

\usepackage{pdfpages}
% allows the insertion of pdf pages, has to be loaded after xcolor package to avoid option clash


%%%%%%%%%
%% FLOATS
%%%%%%%%%

\usepackage{floatrow}
% loads both rotfloat and float packages, rotfloat loads both rotating and float packages
% should not be loaded WITH any of the internally loaded packages
% provides commands to define new floats of various styles and new environments which are rotated by 90 or 270 degrees
% improves the interface for defining floating objects such as figures and tables
% pro­vides th H float mod­i­fier op­tion
% allows to change the font size inside tables


%%%%%%%%%
%% TABLES
%%%%%%%%%

\usepackage{multirow}
% provides the command needed for spanning rows

% defines column types for colored tables
\newcommand*{\arraycolor}[1]{\protect\leavevmode\color{#1}}
\newcolumntype{A}{>{\columncolor{LimeGreen!50!white}}m{4cm}}
\newcolumntype{B}{>{\columncolor{LimeGreen!50!white}}m{2cm}}
\newcolumntype{C}{>{\columncolor{LimeGreen!50!white}}m{0.7cm}}
\newcolumntype{D}{>{\columncolor{Yellow!50!white}}m{0.5cm}}
\newcolumntype{E}{>{\columncolor{LimeGreen!50!white}}m{6cm}}
\newcolumntype{F}{>{\columncolor{Red!50!white}}m{6cm}}
\newcolumntype{G}{>{\columncolor{LimeGreen!50!white}}m{7cm}}
\newcolumntype{H}{>{\columncolor{LimeGreen}}m{3cm}}

\floatsetup[table]{capposition=top}
% for table captions on top


%%%%%%%%%%
%% FIGURES
%%%%%%%%%%

\usepackage{graphicx}
% allows the importation of external graphics (pdf, png and jpg)

\graphicspath{{img/}}
% sets the path to external graphics

\usepackage{epstopdf}
% converts EPS to PDF

\usepackage{svg}
% allows the automated integration of SVG graphics into LATEX documents

\usepackage{caption}
% provides many ways to customise the captions in floating environments

\usepackage{subcaption}


%%%%%%%%%%%
%% LISTINGS
%%%%%%%%%%%

%\usepackage{minted}
%\usepackage{inconsolata}
%\usemintedstyle{monokai}
%\definecolor{bg}{HTML}{282828}
%\setminted{xleftmargin=10pt, linenos=true, fontsize=\footnotesize, tabsize=4, bgcolor=bg}
%\usepackage{tcolorbox}
%\BeforeBeginEnvironment{minted}{\begin{tcolorbox}[colback=bg, colframe=gray, boxsep=0mm]}%
%	\AfterEndEnvironment{minted}{\end{tcolorbox}}%
%
%% change the size and color of the line numbers
%\renewcommand\theFancyVerbLine{\color{gray}\arabic{FancyVerbLine}}


%%%%%%%%%%%%%%%%%%%%%%%%%%
%% BIBLIOGRAPHY MANAGEMENT
%%%%%%%%%%%%%%%%%%%%%%%%%%

\usepackage[
backend=biber,
defernumbers=true,
style=numeric,
backref=true
]{biblatex}
% processes bibliography information, provides an easy and flexible interface and a good language localization 
\addbibresource{bibliography.bib}
% specifies the bibtex data file

\usepackage{csquotes}
% provides advanced facilities for inline and display quotations
% ensures that quoted texts are typeset according to the rules of the main language when using babel or polyglossia with biblatex


%%%%%%%%%%%%%%
%% PAGE LAYOUT
%%%%%%%%%%%%%%

% \usepackage{showframe}
% renders a frame marking the margins of the document

\usepackage[
a4paper, % page size (default is letter, USA standard)
portrait, % page orientation (default is portrait)
top=2.5cm, 
bottom=2.5cm, 
inner=2.3cm, 
outer=2.3cm % specifies the margins
]{geometry}
% allows the customization of document layout

\usepackage{fancyhdr}
% allows fine customization of headers and footers (construction and usage)s
	\setlength{\headheight}{\baselineskip} % height of the header
	\pagestyle{fancy} % page style
	
	\lhead[\course]{\course}
	\chead[]{}
	\rhead[\rightmark]{\rightmark} % current section name printed like "1.6. THIS IS THE SECTION TITLE"
	% headers, [<even output>]{<odd output>}
	
	\lfoot[\thepage]{}
	\cfoot[]{}
	\rfoot[]{\thepage}
	% footers, [<even output>]{<odd output>}

\usepackage{lastpage}
% provides with internal link to last page (\pageref{LastPage})

\setcounter{secnumdepth}{5} % determines up to what level the sectioning titles are numbered
\setcounter{tocdepth}{4} % determines to which level the sectioning commands are printed in the ToC

\setlength{\parindent}{0 cm} % deletes indentation at each new paragraph
\numberwithin{equation}{section} % section numerotation of equation 
\numberwithin{figure}{section} 
\numberwithin{table}{section} 
%\numberwithin{listing}{section}
